\chapter{Introduction}

\section{Forth, the Language}

\subsection{Why Forth?}

Forth is a unique language. What is so special about it? It is a small, low-level language, which can easily be extended to a high-level, domain-specific language that does anything you want it to do. Compared to C64 Basic, Forth is more attractive in almost every way. It is a lot faster, more memory effective, and more powerful.

Compared to C, the nice thing about Forth is that you can run the full development environment on your C64,
with text editor, compiler, assembler and debugger. It makes for a more interactive and fun experience than running a cross-compiler on PC.

For a Forth introduction, please refer to the excellent
\href{http://www.forth.com/starting-forth/}{Starting Forth} by Leo Brodie. As a follow-up, I
recommend \href{http://thinking-forth.sourceforge.net/}{Thinking Forth} by the same author.

\subsection{Comparing to other Forths}

There are other Forths for C64, most notably Blazin' Forth. Blazin' Forth is excellent, but durexForth has some advantages:

\begin{itemize}
\item durexForth uses text files instead of a custom block file system.
\item durexForth is smaller.
\item durexForth is faster.
\item durexForth can boot from cartridge.
\item durexForth fully implements the Forth 2012 core standard.
\item The durexForth editor is a vi clone.
\item durexForth is open source (available at \href{https://github.com/jkotlinski/durexforth}{Github}).
\end{itemize}

\section{Contents}

durexForth is packaged as a 16-kByte cartridge file and a system disk image. Using the cartridge is equivalent to loading durexForth from disk, but of course the cartridge boots a lot faster. The disk image also contains optional Forth modules that can be loaded at will.

\section{Appetizers}

Some demonstration files are included on the disk as appetizers.

\subsection{Graphics}

The gfxdemo package demonstrates the high-resolution graphics, with some examples adapted from the book "Step-By-Step Programming C64 Graphics" by Phil Cornes.
Show the demos by entering:

\texttt{include gfxdemo}

When a demo has finished drawing, press any key to continue.

\subsection{Fractals}

The fractals package demonstrates turtle graphics by generating fractal images. Run it by entering:

\texttt{include fractals}

When an image has finished drawing, press any key to continue.

\subsection{Music}

The mmldemo package demonstrates the MML music capabilities. To play some music:

\texttt{include mmldemo}

\subsection{Sprites}

The sprite package adds functionality for defining and displaying sprites. To run the demo:

\texttt{include spritedemo}

Exit the demo by pressing any key.
