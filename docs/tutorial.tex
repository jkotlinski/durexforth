\chapter{Tutorial}

\section{Interpreter}

Start up durexForth. When loaded, it will greet you with a blinking yellow cursor, waiting for your input. You have landed in the interpreter!

Let's warm it up a little. Enter \texttt{1} (followed by return). You have now put a digit on the stack. This can be verified by the command \texttt{.s}, which will print the stack contents without modifying it. Now enter \texttt{.} to pop the digit and print it to screen, followed by \texttt{.s} to verify that the stack is empty.

Let's define a word \texttt{bg!} for setting the border color\ldots

\begin{verbatim}
: bg! $d020 c! ;
\end{verbatim}

Now try entering \texttt{1 bg!} to change the border color to white.
Then, try changing it back again with \texttt{0 bg!}.

\section{Editor}

The editor (fully described in chapter \ref{editor}) is convenient for editing larger pieces of code. With it, you keep an entire source file loaded in RAM, and you can recompile and test it easily.

Start the editor by typing \texttt{v}. You will enter the red editor screen. To enter text, first press \texttt{i} to enter insert mode. This mode allows you to insert text into the buffer. You can see that it's active on the \texttt{I} that appears in the lower left corner. This is a good start for creating a program!

Now, enter the following lines\ldots

\begin{verbatim}
: flash begin 1 $d020 +! again ; flash
\end{verbatim}

\ldots and then press $\leftarrow$ to leave insert mode.
Press \textsc{F7} to compile and run. If everything is entered right, you will be facing a wonderful color cycle.

When you finished watching, press \textsc{RESTORE} to quit your program, then enter \texttt{v} to reopen the editor.

\section{Assembler}

If you want to color cycle as fast as possible, it is possible to use the durexForth assembler to generate machine code. \texttt{code} and \texttt{;code} define a code word, just like \texttt{:} and \texttt{;} define Forth words. Within a code word, you can use assembler mnemonics.

\begin{verbatim}
code flash
here ( push current addr )
$d020 inc,
jmp, ( jump to pushed addr )
;code
flash
\end{verbatim}

Alternatively, it is possible to use inline assembly within regular Forth words:

\begin{verbatim}
: flash begin [ $d020 inc, ] again ;
flash
\end{verbatim}

Note: As the \texttt{x} register contains the parameter stack depth, it is important that your assembly code leaves it unchanged.

\section{Console I/O Example}

This piece of code reads from keyboard and sends back the chars to screen:

\begin{verbatim}
: foo key emit recurse ;
foo
\end{verbatim}

\section{Avoiding Stack Crashes}

durexForth should be one of the fastest and leanest Forths for the C64. To achieve this, there are
not too many niceties for beginners. For example, compiled code has no checks for stack overflow
and underflow. This means that the system may crash if you do too many pops or pushes. This is not
much of a problem for an experienced Forth programmer, but until you reach that stage, handle the
stack with care.

\subsection{Commenting}

One helpful technique to avoid stack crashes is to add comments about stack usage.
In this example, we imagine a graphics word "drawbox" that draws a black box.
\texttt{( color -- )} indicates that it takes one argument on stack, and on exit it should
leave nothing on the stack. The comments inside the word (starting with \pounds) indicate what the stack
looks like after the line has executed.

\begin{alltt}
: drawbox ( color -- )
10 begin dup 20 < while \pounds color x
10 begin dup 20 < while \pounds color x y
2dup \pounds color x y x y
4 pick \pounds color x y x y color
blkcol \pounds color x y
1+ repeat drop \pounds color x
1+ repeat 2drop ;
\end{alltt}

Once the word is working as supposed, it may be nice to again remove the comments, as
they are no longer very interesting to read.
\par{}\medskip{}
Note: The Forth standard defines the backslash (\textbackslash) as the line comment character, but the C64 lacks a real backslash. Moreover, ASCII \textbackslash{} and PETSCII \pounds{} both map to \${}5c. Therefore, the \pounds{} character is used as a substitution on the C64.

\subsection{Stack Checks}

Another useful technique during development is to check at the end of your main loop
that the stack depth is what you expect it to. This will catch stack underflows
and overflows.

\begin{verbatim}
: mainloop begin
( do stuff here... )
depth abort" depth not 0"
again ;
\end{verbatim}

\section{Configuring durexForth}

\subsection{Stripping Modules}

By default, durexForth boots up with these modules pre-compiled in RAM:

\begin{description}
    \item[asm] The assembler. (Required and may not be stripped.)
    \item[format] Numerical formatting words. (Also required.)
    \item[wordlist] Wordlist manipulation. (Required by some modules.)
    \item[labels] Assembler labels.
    \item[doloop] Do-loop words.
    \item[sys] System calls.
    \item[debug] Words for debugging.
    \item[ls] List disk contents.
    \item[require] The words require and required.
    \item[v] The text editor.
\end{description}

To reduce RAM usage, you may make a stripped-down version of durexForth. Do this by following these steps:

\begin{enumerate}
\item Issue \texttt{---modules---} to unload all modules, or \texttt{---editor---} to unload the editor only.
\item One by one, load the modules you want included with your new Forth. (E.g. \texttt{include labels})
\item Save the new system with e.g. \texttt{save-forth acmeforth}.
\end{enumerate}

\subsection{Custom Start-Up}

You may launch a word automatically at start-up by setting the variable \texttt{start} to the execution token of the word.  Example: \texttt{' megademo start !} To save the new configuration to disk, type e.g. \texttt{save-forth megademo}.

When writing a new program using a PC text editor, it is practical to configure durexForth to compile and execute the program at startup. That can be set up using the following snippet:

\begin{verbatim}
$7000 value buf
: go buf s" myprogramfile" buf
loadb buf - evaluate ;
' go start !
save-forth @0:durexforth
\end{verbatim}

\subsection{Turn-key Operation}

Durexforth offers utilities to save your program in a turn-key fashion by including the \texttt{turnkey} module.

Programs can be saved in a compacted state using \texttt{save-pack}. These programs are stored with 32 bytes between \texttt{here} and \texttt{latest}. When they are first loaded, they will restore the header space to its original \texttt{top}.

If you have developed a program that has no further need of the interpreter, you can eliminate the dictionary entirely when saving with \texttt{save-prg}. This allows your program to use memory down to \texttt{here} plus 32 bytes for safety. 

\section{How to Learn More}

\subsection{Internet Resources}

\subsubsection{Books and Papers}

\begin{itemize}
\item \href{http://www.forth.com/starting-forth/}{Starting Forth}
\item \href{http://thinking-forth.sourceforge.net/}{Thinking Forth}
\item \href{http://www.bradrodriguez.com/papers/}{Moving Forth: a series on writing Forth kernels}
\item \href{http://www.csbruce.com/~csbruce/cbm/transactor/v7/i5/p058.html}{Blazin' Forth --- An inside look at the Blazin' Forth compiler}
\item \href{http://www.drdobbs.com/architecture-and-design/the-evolution-of-forth-an-unusual-langua/228700557}{The Evolution of FORTH, an unusual language}
\item \href{http://galileo.phys.virginia.edu/classes/551.jvn.fall01/primer.htm}{A Beginner's Guide to Forth}
\end{itemize}

\subsubsection{Other Forths}

\begin{itemize}
\item \href{http://www.colorforth.com/cf.html}{colorForth}
\item \href{http://www.annexia.org/forth}{JONESFORTH}
\item \href{http://colorforthray.info/}{colorForthRay.info --- How\_to: with Ray St. Marie}
\end{itemize}

\subsection{Other}

\begin{itemize}
\item \href{https://github.com/jkotlinski/durexforth}{durexForth source code}
\end{itemize}
