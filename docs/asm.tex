\chapter{Assembler}

\section{Introduction}

DurexForth features a simple but useful 6510 assembler with support for branches and labels. Assembly code is typically used within a \texttt{code} word, as in the tutorial example:

\begin{verbatim}
code flash
here        ( push current addr )
d020 inc,   ( inc $d020 )
jmp,        ( jump to pushed addr )
;code
\end{verbatim}

It is also possible to inline assembly code into a regular Forth word, as seen in the tutorial:

\begin{verbatim}
: flash begin [ d020 inc, ] again ;
\end{verbatim}

\section{Variables}

DurexForth has a few variables that are specifically meant to be used within code words.

\begin{description}
    \item[lsb ( -- addr )] Points to the top of the LSB parameter stack. Together with the \texttt{x} register, it can be used to access stack contents.
    \item[msb ( -- addr )] Points to the top of the MSB parameter stack. Together with the \texttt{x} register, it can be used to access stack contents.
    \item[w ( -- addr )] A zero-page cell that code words may use freely as work area.
    \item[w2 ( -- addr )] Second zero-page work area cell.
    \item[w3 ( -- addr )] Third zero-page work area cell.
\end{description}

Example usage of \texttt{lsb} and \texttt{msb}:

\begin{verbatim}
code + ( n1 n2 -- sum )
clc,           ( clear carry )
lsb 1+ lda,x   ( load n1 lsb )
lsb adc,x      ( add n2 lsb )
lsb 1+ sta,x   ( store to n1 lsb )
msb 1+ lda,x   ( load n1 msb )
msb adc,x      ( add n2 msb )
msb 1+ sta,x   ( store to n2 msb )
inx,           ( drop n2; n1 = sum )
;code
\end{verbatim}

\section{Branches}

The assembler supports forward and backward branches. These branches cannot overlap each other, so their usage is limited to simple cases.

\begin{description}
    \item[+branch ( -- addr )] Forward branch.
    \item[:+ ( addr -- )] Forward branch target.
    \item[:- ( -- addr )] Backward branch target.
    \item[-branch ( addr -- )] Backward branch.
\end{description}

Example of a forward branch:

\begin{verbatim}
foo lda,
+branch beq,
bar inc, :+
\end{verbatim}

Example of a backward branch:

\begin{verbatim}
:- d014 lda, f4 cmp,#
-branch bne,
\end{verbatim}

\section{Labels}

The \texttt{labels} module adds support for more complicated flows where branches can overlap freely. These branches are resolved by the \texttt{;code} word, so it is not possible to branch past it.

\begin{description}
    \item[@: ( n -- )] Creates the assembly label n, where n is a number in range [0, 255].
    \item[@@ ( n -- )] Compiles a branch to the label n.
\end{description}

Example:

\begin{verbatim}
code checkers
7f lda,# 0 ldy,# 'l' @:
400 sta,y 500 sta,y
600 sta,y 700 sta,y
dey, 'l' @@ bne, ;code
\end{verbatim}
