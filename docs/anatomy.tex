\chapter{Word Anatomy}

\section{Inspecting a Word}

Let us define a word and see what it gets compiled to.

\begin{verbatim}
: bg $d020 c! ;
\end{verbatim}

Information about the word is split into two areas of memory, the dictionary and the code/data space. Code and data are placed in an upward-growing segment starting at \$801, and the dictionary grows downward from \texttt{top}, default \$6fff. \texttt{latest} points to the last dictionary record. A dictionary record consists of a counted string with flags, and an execution token (XT).

\section{Dictionary}
To inspect the dictionary entry, type \texttt{latest dump}. You should see something like this:

\begin{alltt}
6228  \textcolor{red}{02 42 47 fd 39} 28 39 01 \textcolor{red}{.bg.9}(9.
6230  65 6a 37 09 4d 41 49 4e vJ7.main
...
\end{alltt}

The first byte, "02", is the length of "bg" name string. After that, the string "bg" follows. (42 = 'b', 47 = 'g'). The final two bytes contain the execution token of \texttt{bg}, starting at \$39fd.

The name length byte is also used to store special attributes of the word. Bit 7 is "immediate" flag, which means that the word should execute immediately instead of being compiled into word definitions. ("(" is such an example of an immediate word that does not get compiled.) Bit 6 is "hidden" flag, which makes a word unfindable. Bit 5 is the "no-tail-call-elimination" flag, which makes sure that tail call elimination (the practice of replacing jsr/rts with jmp) is not performed if this word is the jsr target. Since bg does not have these flags set, bits 7-5 are all clear.

\section{Code}
We saw that the "bg" execution token is \$39fd. To inspect the code, type \texttt{\$39fd dump} or \texttt{latest @ dump}.

\begin{alltt}
39fd  \textcolor{red}{20 15 11 20 d0 4c 0e 09  .. Pl..}
6230  65 6a 37 09 4d 41 49 4e vJ7.main
...
\end{alltt}

The code section contain pure 6502 machine code.

\begin{description}
\item[20 15 11 ( jsr \$1115 )] \$1115 is the address of the \texttt{lit} code. \texttt{lit} copies the two following bytes to parameter stack.
\item[20 d0 ( \$d020 )] The parameter to the \texttt{lit} word. When executed, \texttt{lit} will add \$d020 to the parameter stack.
\item[4c 0e 09 ( jmp \$90e )] \$90e is the address of the \texttt{c!} code.
\end{description}

\subsection{Tail call elimination}

The final call to the \texttt{c!} code is a \texttt{jmp} call, while the \texttt{lit} code is called with a \texttt{jsr}. When possible, Durexforth will replace the final \texttt{jsr} followed by \texttt{rts} with a \texttt{jmp}, knowing that the called word or some word it jumps to will execute \texttt{rts}. Some instances cannot benefit from this "tail call elimination", for example if the last word before the \texttt{jsr} is an immediate.

